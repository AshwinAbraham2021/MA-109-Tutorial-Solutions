\documentclass{article}
\usepackage[utf8]{inputenc}
\usepackage{amsmath}
\usepackage{xcolor}
\title{Tutorial 2 Solutions}
\author{Arhaan Ahmad}
\date{November 2022}

\begin{document}
        \maketitle
                Note that the square brackets in the question are not meant to be anything like the greatest integer function. They are just brackets.
        \begin{enumerate}
                \item \begin{enumerate}
                        \item This is false. Take $f(x) = x^2$ and $c=0$, and 
                                $g(x) = \begin{cases}
                                        \frac{1}{x^2} & \text{ if } x \neq 0\\
                                        0 & \text{ if } x = 0\\
                                \end{cases}$. 
                                We have 
                                $f(x)g(x) = \begin{cases}
                                        1 & \text{ if } x \neq 0\\
                                        0 & \text{ if } x = 0\\
                                \end{cases}$. 
                                So $\lim_{x\to 0} f(x)g(x) = 1 \neq 0$
                         \item If $g(x)$ is bounded, let $\max |g(x)| = M$. 
                                 Since $\lim_{x\to c} f(x) = 0$, we have that 
                                 $$\forall \epsilon > 0, \exists \delta > 0, \text{ such that } |f(x) - 0 | < \epsilon \text{ when } 0 < |x-c| < \delta$$


                                 For a particular epsilon $\epsilon_0$, let $\delta_0$, be the delta that is such that 
                                 $|f(x) - 0 | < \frac{\epsilon_0}{M} \text{ when } 0 < |x-c| < \delta_0$

                                 Now, whenever $0 < |x-c| < \delta_0$, 
                                 $$|f(x)g(x) - 0| = |f(x)g(x)| \leq |f(x)| \times M < \frac{\epsilon_0}{M} \times M = \epsilon_0$$
                                 
                                 Hence, the limit of $\lim_{x\to c}f(x)g(x)$ exists and is equal to $0$.
                         \item Let $\lim_{x \to c} g(x)$ be L. We have that 

                                 $$\exists \delta_0 > 0 \text{ such that } |g(x) - L| < |L|\text{ when } 0 < |x-c| < \delta_0$$
                                 $$\implies |g(x)| - |L| < |L| \implies |g(x)| < 2|L| \text{ when } 0 < |x-c| < \delta_0$$

                                 For a particular epsilon $\epsilon_1$, let $\delta_1$, be the delta that is such that 
                                 $|f(x) - 0 | < \frac{\epsilon_1}{2|L|} \text{ when } 0 < |x-c| < \delta_1$

                         Now, $\delta_3 = \min(\delta_0, \delta_1)$ is such that  whenever $0 < |x-c| < \delta_3$, 
                                 $$|f(x)g(x) - 0| = |f(x)g(x)| \leq |f(x)| \times 2|L| < \frac{\epsilon_1}{2|L|} \times 2|L| = \epsilon_1$$

                \end{enumerate}
        \item Let $\lim_{x\to \alpha}f(x) = L$
                $$|f(\alpha+h) - f(\alpha-h) - 0| = |f(\alpha+h) - L - (f(\alpha-h) - L)| < |f(\alpha+h)-L| + |f(\alpha-h) - L|$$

                Since $\lim_{x\to \alpha}f(x) = L$
                $\forall \epsilon, \exists \delta  \text{ such that } |f(x) - L| < \frac{\epsilon}{2}  \text{ when } 0 < |x-\alpha| < \delta $
                Rewriting $h = x-\alpha$, we get
                $\forall \epsilon, \exists \delta_0  \text{ such that } |f(\alpha + h) - L| < \frac{\epsilon}{2}  \text{ when } 0 < |h| < \delta_0 $
                Again rewriting with $h = \alpha-x$, we get 
                $$
                \forall \epsilon, \exists \delta_1  \text{ such that } |f(\alpha - h) - L| < \frac{\epsilon}{2}  \text{ when } 0 < |h| < \delta_1
                $$
                From these two statements, we get that
                $$\forall \epsilon, \exists \delta  \text{ such that } |f(\alpha-h) - L| + |f(\alpha - h) - L| < \epsilon  \text{ when } 0 < |h| < \delta = \min(\delta_1, \delta_2)$$
                $$
                \implies \forall \epsilon, \exists \delta  \text{ such that } |f(\alpha-h) - f(\alpha - h)|< \epsilon  \text{ when } 0 < |h| < \delta = \min(\delta_1, \delta_2)
                $$

                One could have also argued the same using algebra of limits on functions.

                The converse of the result would be that if $\lim_{h\to 0} f(\alpha+h) - f(\alpha-h) = 0$, then $\lim_{x\to \alpha}f(x)$ exists.
                A counterexample to this would be 
                $$f(x) = \begin{cases}
                        \frac{1}{x^2} & \text{ for } x \neq 0\\
                        
                        0  & \text{ for } x =  0\\
                \end{cases}$$
                Take $\alpha = 0$, $f(0+h)-f(0-h) = 0$ for all $h$, since this is an even function, but $\lim_{x\to 0} f(x)$ does not exist.
        \item \begin{enumerate}
                \item 
                        $$f(x) = \begin{cases}
                                \sin \frac{1}{x}  &\text{ if } x\neq 0 \\
                                 0 &\text{ if } x= 0 \\
                            \end{cases}$$

                            In the domain $(0, \infty)$, both $\sin(x)$ and $\frac{1}{x}$ are continuous, hence so is their composition. Same for the domain $(-\infty, 0)$. The only problem is at $x = 0$. The function, in fact, is not continuous at $x=0$. To show this, we'll use the sequential definition of a limit:

                            The limit of a function $f$ at $a$ exists and is equal to $L$ if and only if for every sequence ${x_n} \to a$, $\lim_{n\to \infty}f(x_n) = L$

                            Take ${x_n} = \frac{1}{2n\pi}$ and $y_n = \frac{1}{(2n+\frac{1}{2})\pi}$. We have that $\sin{x_n} = 0$ and $\sin{y_n} = 1$ for all n $\in \mathbf{N}$
                            Thus, we have two sequences tending to $0$, such that $f(x_n)$ and $f(y_n)$ tend to different values, hence the limit of $\sin \frac{1}{x}$ does not exist at $x=0$
                    \item 
                        $$f(x) = \begin{cases}
                                x\sin \frac{1}{x}  &\text{ if } x\neq 0 \\
                                 0 &\text{ if } x= 0 \\
                            \end{cases}$$

                            Here again it is easy to argue continuity in the domains $(0, \infty)$ and $(-\infty, 0)$, using the fact the composition and products of continuous functions is continuous. At $x=0$, we need to find the limit and show that it is equal to $f(0) = 0$.


                            $|x\sin \frac{1}{x}| = |x||\sin \frac{1}{x}| < |x|$, since the range of $\sin$ is $[-1, 1]$

                            Hence,
                            $$\forall \epsilon>0, \, |x\sin \frac{1}{x} - 0| < |x| < \epsilon  \text{ when } 0 < |x-0| < \epsilon$$
                            Seeing that we can just set $\delta = \epsilon$ always, we get that $\lim_{x \to 0} f(x) = 0 = f(0)$. Hence the function is continuous for all reals.
                    \item Note that the pieces of the function in $(1,2)$ and $(2,3)$ are continuous. So there can only be a discontinuity at $x = 2$
                            The right hand and left hand limits at $x = 2$ can both be easily calculated to be $2$, however, $f(2) = 1$. Hence, the function is not continuous at $x=2$
        \end{enumerate}
\item We are given that $f(x+y) = f(x) + f(y)$. Functions with this property are also known as linear functions.
        Set $x=y=0$ to get $f(0) = 2f(0) \implies f(0) = 0$

$$f(c+h) = f(c) + f(h) \implies f(c+h)-f(c) = f(h) = f(h)-f(0)$$
    Since $f$ is continuous at $0$
    $$ \forall \epsilon \exists \delta  \text{ such that }|f(h)-0| < \epsilon  \text{ when } 0 < h < \delta$$
    $$\implies \forall \epsilon \exists \delta  \text{ such that } |f(c+h) - f(c)| = |f(h)| < \epsilon\text{ when } 0 < h < \delta$$
    $\implies$ f is continous for all $c \in \mathbf{R}$    

        \end{enumerate}
\end{document}
